%
% sdd.tex contains the Software/Database Design Document for the
%     Software Development Folder for the eCommerce CMS project for:
% CMSI 402
% Loyola Marymount University
%
% By Andrew Won
%

% ~~~~~~~~~~~~~~~~~~~~~~~~~~~~~~~~~~~~~~~~~~~~~~~~~~~~~~~~~~~~~~~~~~~~~~~~~~~~~
%  Revision History:
%  -----------------
%
%   Ver      Date      Modified by:  Description of change/modification
%  -----  -----------  ------------  ------------------------------------------
%  1.0.0  19-Mar-2013  A. Won        Initial version/release
%
% ~~~~~~~~~~~~~~~~~~~~~~~~~~~~~~~~~~~~~~~~~~~~~~~~~~~~~~~~~~~~~~~~~~~~~~~~~~~~~

\documentclass{article}
\usepackage{doc}
\usepackage{geometry}
\usepackage{amsmath}
\usepackage{multirow}
\geometry{letterpaper}

\title{eCommerce CMS Software/Database Design Document}
\author{Andrew Won}
\date{March 19, 2013}

\newcommand{\br}{\vspace{2mm}}

\begin{document}

\maketitle
\tableofcontents
\listoftables
\listoffigures

\pagebreak
\setcounter{section}{5}
\section{Software/Database Design Document}

\subsection{Introduction}

This document presents the architecture and detailed design for the eCommerce CMS
software for the CMSI 402 project at Loyola Marymount University.  eCommerce CMS
is a software solution for small business customers looking for
an initial online presence.  The eCommerce CMS application will offer both a
web service for HTTP requests and sets of internet-accessible web pages served
by that web service.

The eCommerce CMS application will offer a Java-based web service which will
allow for HTTP requests to qualified users.  The Java-based service will also
serve two sets of front-end web pages.  Through the first set, small business
users of the application can implement and manage an online business management
and sales solution. Through the second set, customers of the small business will
be able to interact with the small business either through making purchases on
an eCommerce web site or through committing information required for generation
of a proposal.

The eCommerce CMS application is the Computer Software Configuration Item
(CSCI) and consists of three major Computer Software Components (CSC).  The
CSCI is broken down into the back-end CSC, the user front-end CSC, and the
customer front-end CSC.  Each respective CSC can further be broken down to
several Computer Software Units (CSU).  This document will specify some of the
CSU's that are intended in table~\ref{software-hierarchy}, but this list shall
in no way bind or limit the number of CSU's that will be developed within
eCommerce CMS.

\begin{table}
    \begin{tabular}{|c|l|p{7.5cm}|}\hline
        CSCI & CSC & CSU \\\hline\hline
        \multirow{12}{*}{eCommerce CMS}
         & Back-End & Persistent Data Store \\\cline{2-3}
         & Back-End & Report/Proposal Generator \\\cline{2-3}
         & Back-End & Front-End Generator \\\cline{2-3}
         & Back-End & Transaction Processor \\\cline{2-3}
         & User Front-End & Visual Web Page Editor \\\cline{2-3}
         & User Front-End & Web Page Template Editor \\\cline{2-3}
         & User Front-End & Inventory Input/Edit Form \\\cline{2-3}
         & User Front-End & Report/Proposal Requestor \\\cline{2-3}
         & Customer Front-End & Proposal Specification Form \\\cline{2-3}
         & Customer Front-End & Web Store \\\cline{2-3}
         & Customer Front-End & Purchasing/Transaction System \\\hline
    \end{tabular}
    \caption{CSCI, CSC, and CSU Hierarchy}
    \label{software-hierarchy}
\end{table}

\subsubsection{System Objectives}

\subsubsection{Hardware, Software, and Human Interfaces}

\pagebreak
\subsection{Architectural Design}

\subsubsection{Major Software Components}
\label{msc}

\subsubsection{Major Software Interactions}
\label{msi}

\subsubsection{Architectural Design Diagrams}
\label{add}

\pagebreak
\subsection{CSC and CSU Descriptions}

\subsubsection{Class Descriptions}
\label{cd}

\begin{enumerate}
    \item[~\ref{cd}.1 ] \emph{Detailed Class Description 1}
\end{enumerate}

\subsubsection{Detailed Interface Descriptions}
\label{did}

\subsubsection{Detailed Data Structure Descriptions}
\label{ddsd}

\subsubsection{Detailed Design Diagrams}
\label{ddd}

\pagebreak
\subsection{Database Design and Description}

\subsubsection{Database Design ER Diagram}
\label{dded}

\subsubsection{Database Access}
\label{da}

\subsubsection{Database Security}
\label{ds}

\end{document}