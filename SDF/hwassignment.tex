%
% hwassignment.tex contains the Semester Homework Assignment for the
%   eCommerce CMS project for:
% CMSI 402
% Loyola Marymount University
%
% By Andrew Won
%

% ~~~~~~~~~~~~~~~~~~~~~~~~~~~~~~~~~~~~~~~~~~~~~~~~~~~~~~~~~~~~~~~~~~~~~~~~~~~~~
%  Revision History:
%  -----------------
%
%   Ver      Date      Modified by:  Description of change/modification
%  -----  -----------  ------------  ------------------------------------------
%  1.0.0  14-Mar-2013  A. Won        Initial version
%
% ~~~~~~~~~~~~~~~~~~~~~~~~~~~~~~~~~~~~~~~~~~~~~~~~~~~~~~~~~~~~~~~~~~~~~~~~~~~~~

\documentclass{article}
\usepackage{doc}
\usepackage{geometry}
\usepackage{amsmath}
\usepackage{float}
\usepackage{sectsty}
\geometry{letterpaper}

\title{CMSI 402 Semester Assignment - ValuJet, System Failures, and Computer Software Engineering}
\author{Andrew Won}
\date{March 14, 2013}

\newcommand{\br}{\vspace{2mm}}
\sectionfont{\large}

\begin{document}

\maketitle

\section{``The Lessons of ValuJet 592'' by William Langewiesche}

There are always mistakes and problems that may arise in the course of
engineering, but few things are as tragic as when lives are lost as a result of
avoidable mistakes.  ``The Lessons of ValuJet 592'' by William Langewiesche
follows the crash, the investigation, and provides insight that might be gained
from ValuJet 592, a twin-engine McDonnell Douglas DC-9 that crashed on May 11,
1996, killing all 110 people on board.  By reviewing
the nature and origin of past problems which led to this accident we can learn
lessons to apply to our future software development.

The article begins with a description of the crash by Walton Little, an
observer who happened to be fishing near the site of the crash.  Mr. Little
reported that, ``There was no smoke, no strange engine noise, no debris in the
air, no dangling materials or control surfaces, no apparent deformation of the
airframe, and no areas that appeared to have missing panels or surfaces,'' before
the plane flew straight into the ground.  Little's account of what was not
present is significant because it rules out many of the typical explanations
involved in plane crashes.  Without any evidence of an engineered problem it
was at first difficult to comprehend why ValuJet 592 had flown straight into
the ground.

Langewiesche remarks that even with a reason for the crash having been found and
actions taken by ValuJet and the FAA, it does not mean that the problem is truly
resolved.  The reason why a true resolution is difficult to find in the ValuJet
592 crash is because of the nature of the accident.  Langewiesche breaks airplane
accidents into three types, as outlined in the table below.  The first two of
which are natural to think of, but the third of which is the elusive trouble-maker
that is difficult to address and that some argue we cannot avoid in the future.

\begin{table}[H]
    \centering
    \begin{tabular}{|l|p{9cm}|}\hline
        \multicolumn{2}{|c|}{Types of airplane accidents}\\\hline
        ``Procedural'' & Result from single obvious mistakes \\\hline
        ``Engineered'' & Materials failures that should have been predicted
        by designers or discovered by test pilots. Usually defy understanding
        at first but yield to examination and result in tangible solutions. \\\hline
        ``System'' & Most elusive.  Charles Perrow, a Yale sociologist, calls
        these ``normal'' accidents because they are the ``bastards born of the
        confusion that lies within the complex organizations with which we
        manage our dangerous technologies.'' \\\hline
    \end{tabular}
\end{table}

The National Transportation Safety Board (NTSB) quickly arrived on the site after the
crash and began their investigation.  Shortly after beginning the investigation
they realized that there may have been an explosion resulting in hazardous materials,
and they may have even realized that it may have originated in the forward cargo
hold, where there was no fire detection or extinguishing systems.  Paperwork indicated
that the forward cargo hold had been loaded with ValuJet ``company material,''
which on this incident happened to be a combination of three tires and five
cardboard boxes of old oxygen generators.  This may have raised some alarms
because oxygen generators should not have been on this flight.

The oxygen generators found in the forward cargo hold of ValuJet 592 were
chemical oxygen generators removed from planes that ValuJet had been
renovating.  These chemical oxygen generators were commonly used in aircraft
to provide oxygen to passengers in the event of sudden loss of pressure in the
cabin of an aircraft.  In the event of a sudden pressure loss, face masks would
fall from the ceiling of the plane and passengers could begin a flow of oxygen
to the masks by pulling on the tube, which would trigger the chemical process
by which oxygen could be generated and delivered to the passengers.  The danger
with these particular oxygen
generators were that they were exothermic- they generated heat outside of the
containers, and were a fire hazard.  This fire hazard was a known danger that
was not treated properly.

The danger posed by the oxygen generators started when mechanics working for
SabreTech, a company routinely hired by ValuJet and many other airlines to do
maintenance and work on airplanes, were renovating three MD-80 airplanes recently
purchased by ValuJet.  Mechanics working for SabreTech removed the oxygen
generators from the airplanes and placed them into cardboard boxes, but failed to
place the required plastic safety caps over the firing pins of the oxygen
generators.  The lack of safety caps meant that these exothermic oxygen
generators could inadvertently fire and pose a threat of fire.  According to the
article, the reason why the safety caps were not used was because ``no one had
such caps, or cared much about finding them.''  SabreTech had failed to provide
the necessary materials to do the job properly, but the mechanics signed off
certifying that safety caps had been placed properly anyways.

The oxygen generators that had been placed into cardboard boxes sat around at
SabreTech for awhile without really being dealt with.  When they finally made
their way to the shipping department at SabreTech, a manager noticed them and
told a shipping clerk to get rid of them to make sure the site was clean for
an upcoming inspection from a potential client.  The shipping clerk secured the
boxes and then asked the receiving clerk to make out a shipping ticket with
``oxygen canisters --empty'' written on them.  They then sat around for another
couple days.  Finally, on May 11, 1996, the boxes containing the oxygen
generators, now incorrectly marked as being empty, were driven to the airport
for ValuJet Flight 592.

When the oxygen generators that SabreTech had removed were driven to the airport,
the ValuJet ramp agent defied federal regulation and accepted the shipment.  ValuJet
was not allowed to carry any item that contained hazardous materials because they
had not been licensed, and even empty oxygen generators were known to discharge
toxic residue.  Regardless, the ValuJet ramp agent accepted the shipment and
together with Richard Hazen, the copilot, who also should have known better than
to accept the oxygen generators, determined that the boxes of oxygen
generators and three spare airplane tires would be placed in the forward cargo
hold.

It is believed that the heat generated by inadvertently fired oxygen generators
was more than adequate to catch the cardboard and spare tires on fire, and that the
cabin was quickly filled with flames and toxic, black smoke.  While the pilot and
copilot should have had oxygen masks that would have protected them from the
toxic smoke, they either did not get the masks on in time to avoid being
poisoned by the smoke or the fire may have spread to the cockpit.  Whatever the
reason, the pilots were unable to turn the plane back towards Miami International
Airport, from which it had taken off, and the plane went into a dive into the
Florida Everglades, despite there being no damage to the outside of the plane.

A tragic accident had occurred that could neither be blamed on a failure of
engineering nor blamed on a single obvious mistake.  The ValuJet 592 crash was
the result of several errors that culminated into a horrific accident and the
loss of several lives.

David Hinson, then administrator of the FAA, came out and made an announcement
that flying on ValuJet was safe and that consumers should not be concerned.  However,
within a few days it became known that inspectors with the FAA had been
concerned about ValuJet because of a ``disproportionate number of infractions''
and because the company had been growing too quickly without the procedures or
people to maintain the requisite level of safety.  Upon review after the crash
it became apparent that ValuJet may have warranted being shut down prior to the
crash, and David Hinson drew criticism for not having taken action that may have
prevented the accident.  Five weeks after the accident ValuJet was permanently
grounded.

The article notes that Charles Perrow blames this type of accident where failures
occurred at many levels on a combination of ``tight coupling'' and ``interactive
complexity,'' by which he means that many elements that are involved are ``linked
in multiple and often unpredictable ways.''  This complexity in the relationship
of these complex systems allows the failure of one part to ``coincide with the
failure of an entirely different part,'' which can then cascade to additional
entirely different parts due to the ``tight coupling.''  The complexity and
rigidity of the airline industry had done its work on ValuJet 592, and the
resulting ``tangle of confusion'' had led to the tragic accident.

Langewiesche notices that the work order from ValuJet called for ``expired'' oxygen
generators to be removed, and that those generators which had not been
``expended'' should have a plastic cap placed on their firing pin.  Langewiesche
notices the ambiguity in differentiating between these two words, and makes an
argument that this technical ``engineerspeak,'' was partly to blame for the
accident.  In addition to the technical jargon, Langewiesche sarcastically notes
that an alternate way that the mechanics could have known how to properly
dispose of the oxygen generators is by opening the ``huge MD-80 maintenance
manual, to chapter 35-22-01,'' line ``h,'' which contained equally ambiguous
``engineerspeak.''  There was excessive documentation that contained technical
jargon not easily understood by temporary mechanics.

But it didn't end with just massive amounts of technical documentation that
needed to be waded through; the industry is inundated by large amounts of
paperwork.  Langewiesche notes that this paperwork may be necessary, but it breeds
a false sense of security.  The presence of documents with signatures asserting
that all things were done properly is no replacement for the actual, proper
execution of safe work.  The lull that came with an industry that thought all
must be safe despite corners being cut, the lack of concern for employees who
seemed to care about their employers as much as their employers cared about them,
a regulatory agency that did little to regulate the industry it was charged to
protect, and a long list of small errors that all aligned into the crash of
ValuJet 592 were tragic failures, but provide lessons to learn for our future.

\section{On the failure of ethics leading to ValuJet 592}

Ethics in terms of engineering is a term that may mean different things to
different people.  Few would argue that when a deliberate action performed with
malicious intent led to the loss of lives an unethical thing had been done.  However,
ethics becomes ambiguous when it is difficult to pinpoint a failure on a single,
purposeful action.  Ethics requires the presence of choices that can be judged as
either right or wrong in view of some coherent philosophy, and in sensitive industries
where lives may be put in danger, special attention should be paid to the actions
of the individuals who make decisions that can impact the safety of other
persons.

Ethics in engineering cannot take the route of mere empirical considerations
as John Stuart Mill's utilitarianism would suggest, nor should it be left to be
determined wholly by reason as Immanuel Kant may argue.  Ethics in engineering
exists to protect the consumers, the society at large, and the community of engineers
themselves.  Ethics in engineering serves a specific purpose to a large group
of people, but circumventing ethical concerns may pose beneficial to some
individuals and organizations.  Because of the necessity of ethics in encouraging
the broader society, ethics in the realm of engineering, here including the airline
industry, should be a function of governing.  Whether governed by an independent
body or by the state government itself, ethics in engineering serves a specific
function of government.  Therefore, the ethics that were violated in causing the
crash of ValuJet 592 do not necessarily have to stem from a singular school of
philosophy, but may be relegated to a set of decisions made by the community in
order to protect itself and its consumers.

One of the first ``ethical'' failures in this regard lies with the creation of
ValuJet itself.  As the company's name may imply, ValuJet existed to provide a
cheaper solution to those wishing to fly, but it isn't possible to focus on
cutting costs and provide the same attention to providing safety.  Langewiesche
notes that ``safety is never first, and it never will be,'' because
of the competitive nature of flying.  It is expensive to fly, but there are
increasing amounts of people who are willing to pay a reduced fare to fly.  This
offers a growing market to airlines who are able to find ways to save money, even
if this means cutting corners.  And that is exactly what ValuJet was doing; when
ValuJet 592 took its fateful flight ValuJet was a rapidly expanding company that
had found many ways to reduce costs by cutting corners.

The corners that were being cut started with its employees.  One of the first
things that many managers learn is that one of the most controllable expenses is
payroll.  It made sense to ValuJet to forgo permanent, full-time employees and
use temporary workers.  Full-time employees require benefits, require leaves of
absences, are protected by laws, require payroll taxes, and have to be paid even
when there is insufficient work to warrant their presence.  The use of temporary
employees allowed ValuJet to save a lot of money, regardless of what consequences
that may bring.  While paying more money to an employee does not necessitate
better, safer work, it is no surprise that a corporate culture which paid little
attention to its employees was doing little about numerous minor safety infractions
that had been appearing.  ValuJet's primary concern was not with empowering its
employees to make safe actions but with saving money wherever money could be saved.

Another ``ethical'' failure existed at the lowest level, the failure of each
individual worker who had contact with the oxygen generators which eventually
landed in the forward cargo hold of ValuJet 592.  While the company and the
industry were guilty for the culture which fostered and ignored a lack of safe
behavior from the individual employees, the employees themselves were willfully
acting within this broken system.  Multiple occurrences of ``pencil-whipping''
and the acceptance of hazardous materials by both a ValuJet ramp agent and copilot
may have been considered a normal course of action to the individuals at the time,
but it is clear in retrospect that these actions were not safe or appropriate.  The
nature of the system accident is such that no single individual can receive blame,
but there are a series of individuals that can share it.  If any one of the many
individuals who had contact with the oxygen generators had thought to speak up
and question the presence of those generators, the ValuJet 592 crash may never
have happened.  Langewiesche and Perrow, however, may argue that an individual
speaking up would do nothing but delay such an accident.

Charles Perrow argues that contrary to Murphy's law, ``what can go wrong usually
goes right.''  Everyone had grown used to taking shortcuts, and as a result it
was natural for everyone who came along and worked with the expired oxygen
generators to ignore the safety threat that they posed.  While an individual
speaking up and stopping the oxygen generators that were destined for ValuJet
592 may have saved that flight, the broken system itself would have eventually
fostered another similar incident just at another place and time.  Because of the
widespread nature of the problem, one cannot stop at recognizing the ``ethical''
failures of the individual and the company, but must go beyond to the broadest
sense.

The crash of ValuJet 592 was a result of multiple failures, but it may perhaps
stem from an ``ethical'' failure on the part of the airline industry as an
aggregate whole.  Langewiesche notes that following deregulation of the airline
industry it made sense to find ways to save money, and that cutting corners had
become a routine practice across the industry.

David Hinson, administrator of the FAA, following the crash, went to Congress
and requested that the FAA's ``dual mandate,'' which included promoting airlines,
be removed.  Langewiesche refers to this as mere theatre, but it is telling
that the federal agency primarily in charge of the safe function of the airlines
was initially charged with promoting the industry itself.  This mandate to promote
the industry seemed contradictory to the requirement of ensuring the safety of
consumers.  It translated into an agency that, along with the NTSB, had little
actual ability to regulate anybody.  The FAA did exist to promote safety, but
with contradicting mandates and not enough people to watch over every airline
employee, it was wholly inadequate to ensure anyone's safety.  The best that the
FAA could do was outline rules and standards for airlines to abide by, and these
led to the large number of processes making up the broad, complex system that
the airline industry had become.

The broad, complex airline industry laid down many procedures to abide by, but
cared little about the actual result of the actions of the employees working
for the airlines.  Paperwork and precise instruction existed covering nearly every
mistake that led to the crash of ValuJet 592, but no one in the industry cared
to think beyond the proper execution of the paperwork and the saving of
costs.  The nature of the airline industry has become a mere process that has
moved away from the very human dangers that are present in the improper execution
of the functions of employees working in the industry.

The ``ethical'' failures of the airline industry, ValuJet, and the employees that
handled the oxygen generators that ended up in the forward cargo hold of ValuJet
592 were many, and can be summed up in many numbers of ways.  But the ultimate
failure of the industry as a whole was a lack of focus on actual safety and a
focus on merely the processes laid down by the FAA and the complex system that
had to be navigated in order to qualify as a ``safe'' airline and make money.

\section{What Computer Software Engineering can learn from ValuJet 592}

When planning, designing, and developing a software product or application, there
are many things that can go wrong.  There are even many software endeavors that
bear consequence on people's lives.  There are even some widely used industry
standards that layer complex safety documentation and checks in place, just like
the airline industry.  While there are many differences between software
engineering and the airline industry, there are still more than enough parallels
to learn lessons from the system failure that happened with ValuJet 592.

The pilot, Captain Candalyn Kubeck with significant experience flying, earned
about \$43,000, and her co-pilot, Richard Hazen with similarly significant
experience, earned about half of that.  The article notes that along with the
pilots, flight attendants, ramp agents, and mechanics were all paid much less
than a more traditional airline would have paid.  ValuJet became notorious within
the industry for its use of temporary employees and independent contractor such that
some began to call it a ``virtual airline.''  All of this success, however, led
to rapid expansion, and some FAA regulators had even begun to become concerned
about whether ValuJet could really maintain quality and all the paperwork required
at the rate that it was growing.

When the oxygen generators that may have caused the ValuJet 592 crash were removed
from the three MD-80 aircraft that ValuJet had purchased, they were done so by
a contractor that ValuJet had hired, SabreTech.  SabreTech was a large firm that
often did this type of work, but the article notes that SabreTech routinely hired
contract mechanics that worked on an as-needed basis.  This resulted in about
three-fourths of the people on the ValuJet project being temporary workers that
had been contracted.  Not only was SabreTech using contracted temporary employees,
but it also wasn't treating them very well.  When the contract deadline was drawing
close, SabreTech had employees working on shifts day and night and even on some
weekends.  SabreTech, like ValuJet, was finding ways to save money by
cutting corners; minimizing payroll, benefits, taxes, and other expenses that come
with hiring full-time employees; and not treating their employees well.

While neither ValuJet nor SabreTech seemed to care about their employees very
much, it is not possible to argue that if either ValuJet or SabreTech paid their
employees more this accident may have been avoided.  That does not mean, however,
that considering the way in which both companies saved money when it came to
personnel should not be taken into consideration.  The fact that both companies
avoided full-time staff and worked with a large number of contracted temporary
workers was a great way to save money, but it was not a great way to get employees
who were willing to work very hard.  Through several steps of the process by which
the oxygen generators which may have driven ValuJet 592 into the ground ended up
on the airplane, it was
apparent that the employees had not taken too much consideration into their
jobs; Langewiesche calls this ``pencil-whipping.''  The bottom line was that the
employees were not motivated to do anything beyond what they had to in order to
receive their paycheck.

Dan Pink at a talk for the Royal Society for the encouragement of Arts,
Manufactures and Commerce, the RSA, mentioned that one way to motivate people
in jobs that are knowledge-based is to, ``pay people enough so
they�re not thinking about money but thinking about their work,'' and that the
three factors that lead to better performance after money is taken ``off the
table'' are autonomy, mastery, and purpose.  His fundamental argument was that
you could not have employees that were motivated to try, work hard, and care
about the results of their work when their primary concerns lie apart from the
actual job.  Computer software engineers play an important role in software
development and are empowered to do great jobs or make critical errors in the
course of their software development.  While it is not necessary to just pay
each employee a lot of money, Pink's argument is that employees should not spend
their time thinking about ways to make more money.  Employees should spend time
caring about the work and the results of the work that they are performing.  It
is not easy, however, to motivate this kind of conern.

Autonomy, mastery, and purpose are hard to foster, and are made even more difficult
by the presence of layers of safety guards and standard procedures designed to
remove autonomy.  Computer software development cannot rely on a single planning
stage in the development of software.  The computer software engineers who
are tasked with developing the mission critical software that lives depend upon
will have little incentive to catch mistakes and do more than the minimum
required of them to perform their task to completion if their only task is to
meet the rigid requirements of an abstract document.

A perfect example of this was an anecdote related by Professor Robert ``B.J.''
Johnson of Loyola Marymount University.  B.J. related a past, large-scale
software development project that he had participated in where his immediate
supervisor was under pressure to meet time deadlines for his project.  B.J. had
created a Java class, a sub-program part of a larger program, that was not of
major importance, but had not implemented any actual functionality in the
class.  When the supervisor came by and asked him how the class was doing he
explained that the class existed and could successfully compile into a working
subprogram, but did not have any functionality.  The supervisor considered the
importance of completing the project on time and instructed B.J. to place the
class with no functionality into the repository of completed software classes
and mark the class as done.  This falsification of completion bears so much
resemblance to the ``pencil-whipping'' that occurred with the ValuJet 592
project that it is hard not to draw the parallels.

The waterfall model of software development is a sequential software development
model commonly used in highly sensitive projects where safety and security are
highly important.  This was modeled after many other engineering methodologies
and relies on front-loaded customer involvement where requirements are clearly
specified to great extent in extensive and complex requirements
specifications.  However, as the anecdote that Professor Johnson related exhibits,
there are times when these detailed requirements specifications fail to translate
into ideally executed software.  The over-specification of the development process
creates rigidity that leads to engineers who live with the mentality that it is
more important to just get the job done than to do it well.  The existence of
detailed requirements documents lulls managers into a false sense of security
and allows them to assume that all aspects of a software may work correctly if
all elements of a requirement specification are checked off.

However, the simple removal of the initial software design process would not by
its own right improve the quality of the code produced.  Robert ``Bob'' Martin,
in his book Clean Code, discusses the importance of customer, or manager, involvement
throughout the development process.  When there are actual stakeholders involved
in the development stage of software, ``pencil-whipping'' is less likely to
occur.  While some may pose the argument that the resources required to actually
have stakeholders review the software produced over short iterations and constantly
review the specification may not be worth whatever benefit it may convey, Martin
argues that companies cannot afford to not have stakeholders review throughout
the development. Front-loading the design process turns the development stage into
the exact mindless job that lacks any motivation that the workers who had encountered
the oxygen generators that were destined for ValuJet 592 had faced.  The presence
of rigid timelines and simple paperwork to ensure success has proven to be
inadequate in ensuring proper execution.




\pagebreak

\bibliography{hwassignment}
%\bibliographystyle{apalike}

\end{document}