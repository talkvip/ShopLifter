%
% hwassignment.tex contains the Semester Homework Assignment for the
%   eCommerce CMS project for:
% CMSI 402
% Loyola Marymount University
%
% By Andrew Won
%

% ~~~~~~~~~~~~~~~~~~~~~~~~~~~~~~~~~~~~~~~~~~~~~~~~~~~~~~~~~~~~~~~~~~~~~~~~~~~~~
%  Revision History:
%  -----------------
%
%   Ver      Date      Modified by:  Description of change/modification
%  -----  -----------  ------------  ------------------------------------------
%  1.0.0  11-Mar-2013  A. Won        Initial version
%
% ~~~~~~~~~~~~~~~~~~~~~~~~~~~~~~~~~~~~~~~~~~~~~~~~~~~~~~~~~~~~~~~~~~~~~~~~~~~~~

\documentclass{article}
\usepackage{doc}
\usepackage{geometry}
\usepackage{amsmath}
\usepackage{float}
\usepackage{sectsty}
\geometry{letterpaper}

\title{CMSI 402 Semester Assignment - ValuJet, System Failures, and Computer Science}
\author{Andrew Won}
\date{March 13, 2013}

\newcommand{\br}{\vspace{2mm}}
\sectionfont{\large}

\begin{document}

\maketitle

% ValuJet Essay Assignment � Write a 5000-word essay on the ValuJet article from
% the Atlantic Magazine, which I will provide for you. This is also a
% "critical thinking" exercise, and is partly an opinion piece, so again there
% is no wrong answer. The essay should be in three parts:
% 1) a synopsis of the article's main points as far as the crash, the
% investigation, and its cause;
% 2) your thoughts on the ethics issues involved, detailing not just what was
% wrong, but why it was wrong and its possible contributions to other potential
% catastrophic system failures; and
% 3) the types of similar or different problems which might arise when planning,
% designing, and developing a software product or application.

\section{``The Lessons of ValuJet 592'' by William Langewiesche}

There are always mistakes and problems that may arise in the course of
engineering, but few things are as tragic as when lives are lost as a result of
avoidable mistakes.  ``The Lessons of ValuJet 592'' by William Langewiesche
follows the crash, the investigation, and provides insight that might be gained
from ValuJet 592, a twin-engine McDonnell Douglas DC-9 that crashed on May 11,
1996, killing all 110 people on board.  By reviewing
the nature and origin of past problems which led to this accident we can learn
lessons to apply to our future software development.

The article begins with a description of the crash by Walton Little, an
observer who happened to be fishing near the site of the crash.  Mr. Little
reported that, ``There was no smoke, no strange engine noise, no debris in the
air, no dangling materials or control surfaces, no apparent deformation of the
airframe, and no areas that appeared to have missing panels or surfaces,'' before
the plane flew straight into the ground.  Little's account of what was not
present is significant because it rules out many of the typical explanations
involved in plane crashes.  Without any evidence of an engineered problem it
was at first difficult to comprehend why ValuJet 592 had flown straight into
the ground.

Langewiesche remarks that even with a reason for the crash having been found and
actions taken by ValuJet and the FAA, it does not mean that the problem is truly
resolved.  The reason why a true resolution is difficult to find in the ValuJet
592 crash is because of the nature of the accident.  Langwiesche breaks airplane
accidents into three types, as outlined in the table below.  The first two of
which are natural to think of, but the third of which is the elusive trouble-maker
that is difficult to address and that some argue we cannot avoid in the future.

\begin{table}[H]
    \centering
    \begin{tabular}{|l|p{9cm}|}\hline
        \multicolumn{2}{|c|}{Types of airplane accidents}\\\hline
        ``Procedural'' & Result from single obvious mistakes \\\hline
        ``Engineered'' & Materials failures that should have been predicted
        by designers or discovered by test pilots. Usually defy understanding
        at first but yield to examination and result in tangible solutions. \\\hline
        ``System'' & Most elusive.  Charles Perrow, a Yale sociologist, calls
        these ``normal'' accidents because they are the ``bastards born of the
        confusion that lies within the complex organizations with which we
        manage our dangerous technologies.'' \\\hline
    \end{tabular}
\end{table}

The National Transportation Safety Board (NTSB) quickly arrived on the site after the
crash and began their investigation.  Shortly after beginning the investigation
they realized that there may have been an explosion resulting in hazardous materials,
and they may have even realized that it may have originated in the forward cargo
hold, where there was no fire detection or extinguishing systems.  Paperwork indicated
that the forward cargo hold had been loaded with ValuJet ``company material,''
which on this incident happened to be a combination of three tires and five
cardboard boxes of old oxygen generators.  This may have raised some alarms
because oxygen generators should not have been on this flight.

The oxygen generators found in the forward cargo hold of ValuJet 592 were
chemical oxygen generators removed from planes that ValuJet had been
renovating.  These chemical oxygen generators were commonly used in aircraft
to provide oxygen to passengers in the event of sudden loss of pressure in the
cabin of an aircraft.  In the event of a sudden pressure loss, face masks would
fall from the ceiling of the plane and passengers could begin a flow of oxygen
to the masks by pulling on the tube, which would trigger the chemical process
by which oxygen could be generated and delivered to the passengers.  The danger
with these particular oxygen
generators were that they were exothermic- they generated heat outside of the
containers, and were a fire hazard.  This fire hazard was a known danger that
was not treated properly.

The danger posed by the oxygen generators started when mechanics working for
SabreTech, a company routinely hired by ValuJet and many other airlines to do
maintenance and work on airplanes, were renovating three MD-80 airplanes recently
purchased by ValuJet.  Mechanics working for SabreTech removed the oxygen
generators from the airplanes and placed them into cardboard boxes, but failed to
place the required plastic safety caps over the firing pins of the oxygen
generators.  The lack of safety caps meant that these exothermic oxygen
generators could inadvertently fire and pose a threat of fire.  According to the
article, the reason why the safety caps were not used was because ``no one had
such caps, or cared much about finding them.''  SabreTech had failed to provide
the necessary materials to do the job properly, but the mechanics signed off
certifying that safety caps had been placed properly anyways.

The oxygen generators that had been placed into cardboard boxes sat around at
SabreTech for awhile without really being dealt with.  When they finally made
their way to the shipping department at SabreTech, a manager noticed them and
told a shipping clerk to get rid of them to make sure the site was clean for
an upcoming inspection from a potential client.  The shipping clerk secured the
boxes and then asked the receiving clerk to make out a shipping ticket and write,
``oxygen canisters --empty'' on the ticket.



\section{On the ethics of ValuJet 592}







Langewiesche notes that ``safety is never first, and it never will be,'' because
of the competitive nature of flying.  It is expensive to fly, but there are
increasing amounts of people who are willing to pay a reduced fare to fly.  This
offers a growing market to airlines who are able to find ways to save money, even
if this means cutting corners.  And that is exactly what ValuJet was doing; when
ValuJet 592 took its fateful flight ValuJet was a rapidly expanding company that
had found many ways to reduce costs by cutting corners.

The pilot, Captain Candalyn Kubeck with significant experience flying, earned
about \$43,000, and her co-pilot, Richard Hazen with similarly significant
experience, earned about half of that.  The article notes that along with the
pilots, flight attendants, ramp agents, and mechanics were all paid much less
than a more traditional airline would have paid.  ValuJet became notorious within
the industry for its use of temporary employees and independent contractor such that
some began to call it a ``virtual airline.''  All of this success, however, led
to rapid expansion, and some FAA regulators had even begun to become concerned
about whether ValuJet could really maintain quality and all the paperwork required
at the rate that it was growing.

When the oxygen generators that may have cause the ValuJet 592 crash were removed
from the three MD-80 aircraft that ValuJet had purchased, they were done so by
a contractor that ValuJet had hired, SabreTech.  SabreTech was a large firm that
often did this type of work, but the article notes that SabreTech routinely hired
contract mechanics that worked on an as-needed basis.  This resulted in about
three-fourths of the people on the ValuJet project being temporary workers that
had been contracted.  Not only was SabreTech using contracted temporary employees,
but it also wasn't treating them very well.  When the contract deadline was drawing
close, SabreTech had employees working on shifts day and night and even on some
weekends.  SabreTech, like ValuJet, was finding ways to save money by
cutting corners; minimizing payroll, benefits, taxes, and other expenses that come
with hiring full-time employees; and not treating their employees well.

While neither ValuJet nor SabreTech seemed to care about their employees very
much, it is not possible to argue that if either ValuJet or SabreTech paid their
employees more this accident may have been avoided.  That does not mean, however,
that considering the way in which both companies saved money when it came to
personnel should not be taken into consideration.  The fact that both companies
avoided full-time staff and worked with a large number of contracted temporary
workers was a great way to save money, but it was not a great way to get employees
who were willing to work very hard.  Through several steps of the process by which
the oxygen generators which may have driven ValuJet 592 into the ground it was
apparent that employees were not taken too much consideration into their
jobs; Langewiesche calls this ``pencil-whipping.''  The bottom line was that the
employees were not motivated to do anything beyond what they had to in order to
receive their paycheck.

Dan Pink at a talk for the Royal Society for the encouragement of Arts,
Manufactures and Commerce, the RSA, mentioned that one way to motivate people
in jobs that are knowledge-based is to, ``pay people enough so
they�re not thinking about money but thinking about their work,'' and that the
three factors that lead to better performance after money is taken ``off the
table'' are autonomy, mastery, and purpose.

\pagebreak

\bibliography{hwassignment}
\bibliographystyle{apalike}

\end{document}