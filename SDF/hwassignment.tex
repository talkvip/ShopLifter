%
% hwassignment.tex contains the Semester Homework Assignment for the
%   eCommerce CMS project for:
% CMSI 402
% Loyola Marymount University
%
% By Andrew Won
%

% ~~~~~~~~~~~~~~~~~~~~~~~~~~~~~~~~~~~~~~~~~~~~~~~~~~~~~~~~~~~~~~~~~~~~~~~~~~~~~
%  Revision History:
%  -----------------
%
%   Ver      Date      Modified by:  Description of change/modification
%  -----  -----------  ------------  ------------------------------------------
%  1.0.0  11-Mar-2013  A. Won        Initial version
%
% ~~~~~~~~~~~~~~~~~~~~~~~~~~~~~~~~~~~~~~~~~~~~~~~~~~~~~~~~~~~~~~~~~~~~~~~~~~~~~

\documentclass{article}
\usepackage{doc}
\usepackage{geometry}
\usepackage{amsmath}
\usepackage{float}
\geometry{letterpaper}

\title{CMSI 402 Semester Assignment - ValuJet, System Failures, and Computer Science}
\author{Andrew Won}
\date{March 13, 2013}

\newcommand{\br}{\vspace{2mm}}

\begin{document}

\maketitle

\abstract{

}

\pagebreak

% ValuJet Essay Assignment � Write a 5000-word essay on the ValuJet article from
% the Atlantic Magazine, which I will provide for you. This is also a
% "critical thinking" exercise, and is partly an opinion piece, so again there
% is no wrong answer. The essay should be in three parts:
% 1) a synopsis of the article's main points as far as the crash, the
% investigation, and its cause;
% 2) your thoughts on the ethics issues involved, detailing not just what was
% wrong, but why it was wrong and its possible contributions to other potential
% catastrophic system failures; and
% 3) the types of similar or different problems which might arise when planning,
% designing, and developing a software product or application.

There are always mistakes and problems that may arise in the course of
engineering, but few things are as tragic as when lives are lost
``The Lessons of ValuJet 592'' by William Langewiesche follows the crash, the
investigation, and provides insight that might be gained from ValuJet 592, a
twin-engine DC-9 that crashed in May of 1996.

Langewiesche notes that ``safety is never first, and it never will be,'' because
of the competitive nature of flying.  It is expensive to fly, but there are
increasing amounts of people who are willing to pay a reduced fare to fly.  This
offers a market to

    \begin{table}[H]
        \centering
        \begin{tabular}{|l|p{8.5cm}|}\hline
            \multicolumn{2}{|c|}{Types of airplane accidents}\\\hline
            ``Procedural'' & Result from single obvious mistakes \\\hline
            ``Engineered'' & Materials failures that should have been predicted by designers or discovered by test pilots. Usually defy understanding at first but yield to examination and result in tangible solutions. \\\hline
            ``System'' & Most elusive. Charles Perrow, a Yale sociologist, calls these ``normal'' accidents because they are the ``bastards born of the confusion that lies within the complex organizations with which we manage our dangerous technologies.'' \\\hline
        \end{tabular}
    \end{table}





Dan Pink at a talk for the Royal Society for the encouragement of Arts,
Manufactures and Commerce, the RSA, mentioned that one way to motivate people
in jobs that are knowledge-based is to, ``pay people enough so
they�re not thinking about money but thinking about their work,'' and that the
three factors that lead to better performance after money is taken ``off the
table'' are autonomy, mastery, and purpose.

\pagebreak

\bibliography{hwassignment}
\bibliographystyle{apalike}

\end{document}