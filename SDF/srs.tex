%
% srs.tex contains the Software Requirements Specification for the
%     Software Development Folder for the eCommerce CMS project for:
% CMSI 402
% Loyola Marymount University
%
% By Andrew Won
%

\documentclass{article}
\usepackage{doc}
\usepackage{geometry}
\usepackage{amsmath}
\usepackage{multirow}
\geometry{letterpaper}

\title{eCommerce CMS Software Requirements Specification}
\author{Andrew Won}
\date{February 4, 2013}

\newcommand{\br}{\vspace{2mm}}

\begin{document}

\maketitle

\abstract{
eCommerce CMS is a software solution for small business customers looking for
an initial online presence.  The eCommerce CMS application will offer both a
web service for HTTP requests and sets of internet-accessible web pages served
by that web service.

This Software Requirements Specification is intended to outline the necessary
outcomes and performance specifications required for eCommerce CMS to be
considered fully functional and deployable.
}

\pagebreak
\tableofcontents
\listoffigures
\listoftables

\pagebreak
\setcounter{section}{4}
\section{Software Requirements Specification}

\subsection{Introduction}

\subsubsection{Identification}

This Software Requirements Specification (SRS) documents the requirements for
the small business utility web service and web site, called eCommerce CMS.

\subsubsection{System Overview}

The eCommerce CMS application will offer a Java-based web service which will
allow for HTTP requests to qualified users.  The Java-based service will also
serve two sets of front-end web pages.  Through the first set, small business
users of the application can implement and manage an online business management
and sales solution. Through the second set, customers of the small business will
be able to interact with the small business either through making purchases on
an eCommerce web site or through committing information required for generation
of a proposal.

The eCommerce CMS application is the Computer Software Configuration Item 
(CSCI) and consists of three major Computer Software Components (CSC).  The 
CSCI is broken down into the back-end CSC, the user front-end CSC, and the 
customer front-end CSC.  Each respective CSC can further be broken down to 
several Computer Software Units (CSU).  This document will specify some of the 
CSU's that are intended in table~\ref{software-hierarchy}, but this list shall 
in no way bind or limit the number of CSU's that will be developed within 
eCommerce CMS.

\begin{table}
    \begin{tabular}{|c|c|c|}\hline
        CSCI & CSC & CSU \\\hline\hline
        \multirow{12}{*}{eCommerce CMS}
         & Back-End & Persistent Data Store (Section~\ref{func-back-end-2}) \\\cline{2-3}
         & Back-End & Report/Proposal Generator (Section~\ref{func-back-end-3}) \\\cline{2-3}
         & Back-End & Front-End Generator (Sections~\ref{func-back-end-4} and~\ref{func-back-end-5}) \\\cline{2-3}
         & Back-End & Transaction Processor (Sections~\ref{func-back-end-6} and~\ref{func-back-end-7}) \\\cline{2-3}
         & User Front-End & Visual Web Page Editor \\\cline{2-3}
         & User Front-End & Web Page Template Editor \\\cline{2-3}
         & User Front-End & Inventory Input/Edit Form \\\cline{2-3}
         & User Front-End & Report/Proposal Requester \\\cline{2-3}
         & Customer Front-End & Proposal Specification Form \\\cline{2-3}
         & Customer Front-End & Web Store \\\cline{2-3}
         & Customer Front-End & Purchasing/Transaction System \\\cline{2-3}
         & Customer Front-End & Product Browser \\\hline
    \end{tabular}
    \caption{CSCI, CSC, and CSU Hierarchy}
    \label{software-hierarchy}
\end{table}

\subsubsection{Document Overview}

This SRS is a minimized form of a typical SRS due to time constraints.  The
remainder of this document is structured as follows.  Section 5.2 contains 
Functional Requirements, Section 5.3 contains Performance Requirements, and 
finally Section 5.4 contains Environmental Requirements.  Following these
Requirements sections there will be an appendix.

Table~\ref{terms} defines key terms that are used within this document.

\begin{table}
    \begin{tabular}{|l|p{11cm}|}\hline
        Term & Definition \\\hline\hline
        Shall & A contractual obligation without which this software shall be considered incomplete. \\\hline
        Should & An optional provision offering guidance for functional requirements. \\\hline
        Will & An optional provision offering guidance for design requirements. \\\hline
        User & Small business owner, or agent thereof, intending to deploy a web service
        and corresponding web sites. \\\hline
        Customer & A customer of the small business owner who intends to use the web service
        or one of the web sites deployed by the user. \\\hline
        Application & The eCommerce CMS software suite outlined in this SRS. \\\hline
        Back-End & The web service offering HTTP endpoints and serving web sites \\\hline
        User Front-End & The collection of web sites that offer creation and management
        functionality to the user. \\\hline
        Customer Front-End & The collection of web sites customized by the user displaying
        a web store or data entry fields for proposal generation.\\\hline
    \end{tabular}
    \caption{Definitions used in this SRS.}
    \label{terms}
\end{table}

\subsection{Functional Requirements}

The application will serve eCommerce web pages allowing inventory data to be
manipulated and transacted with by the user and customer's, respectively.  The
user will have the ability to also manipulate the display of the customer's
front-end.

The Functional Requirements section is segmented into subsections consisting of
one functional requirement each.  These subsections flow sequentially from
back-end, to user front-end, and finally to customer front-end.

\subsubsection{Functional Back-End Requirement 1}
\label{func-back-end-1}

The back-end shall provide a URI contract with a set of HTTP endpoints
to access web service functions for all functions offered by the web service
when hit by an HTTP request.

\subsubsection{Functional Back-End Requirement 2}
\label{func-back-end-2}

The back-end shall provide a persistent data store with saved input data from
the user and from customers to offer invoice, proposal, report, and web store
content generation by the user.

\subsubsection{Functional Back-End Requirement 3}
\label{func-back-end-3}

The back-end shall provide an endpoint that generates either a report or a
proposal based off of the selection and fields provided by the user in the
body of the request.

\subsubsection{Functional Back-End Requirement 4}
\label{func-back-end-4}

The back-end shall provide a user front-end generator that deploys a set of
documents written in standard HTML and allowing for management of the eCommerce
site when deployed by the user.

\subsubsection{Functional Back-End Requirement 5}
\label{func-back-end-5}

The back-end shall provide a customer front-end generator that deploys a set of
documents written in standard HTML with customization based on past
user input when deployed by HTTP endpoint.

\subsubsection{Functional Back-End Requirement 6}
\label{func-back-end-6}

The back-end shall provide a transaction processor, or connection to a 
third-party transaction processor, to accept purchasing by customers
when a customer completes a check-out process.

\subsubsection{Functional Back-End Requirement 7}
\label{func-back-end-7}

The back-end shall provide a transaction processor that logs data pertaining
to either a sale of inventory or a generation of a proposal for use in report
generation, as outlined in section~\ref{func-back-end-3}, when a transaction
is conducted.

\subsubsection{Functional User Front-End Requirement 1}
\label{func-user-front-end-1}

%    [agent] shall <perform> [FUNCTION] with [Performance] <parameter(s)> while
%    [CONDITIONS] <upon input(s)> in accordance with [INTERFACES].

\subsubsection{Functional Customer Front-End Requirement 1}
\label{func-cust-front-end-1}


\subsection{Performance Requirements}

\subsubsection{Performance Requirement 1}
\label{perf-back-end-1}

The back-end shall respond to all front-end page requests through HTTP endpoints
within 2 seconds in accordance with functional back-end
requirement~\ref{func-back-end-1}.

\subsection{Environment Requirements}
\subsubsection{Development Environment Requirements}

Maven, Java

\subsubsection{Execution Environment Requirements}

The back-end is served from a web server or from a hosting solution provider and is
required to have the minimum specifications outlined in table~\ref{server}.  The
server itself requires minimum hardware support, but does require an always-on
network connection through which to serve the service and web pages.  A Java
Development Kit (JDK) of version 1.5 or newer is required to compile and package
the Maven repository that will be used.

The front-end is served as web pages within supported web browsers, as listed in
table~\ref{browsers}.  This does not constitue a complete list of supported web
browsers, but is a brief list of web browsers that are guaranteed to be supported
by the web sites.

\begin{table}
    \centering
    \begin{tabular}{|c|c|}\hline
        Category & Requirement \\\hline\hline
        JDK & 1.5+ \\\hline
        Hard Drive Space & 250mb \\\hline
        Operating System & No requirement \\\hline
        RAM & 512mb \\\hline
    \end{tabular}
    \caption{Server Requirements}
    \label{server}
\end{table}

\begin{table}
    \centering
    \begin{tabular}{|c|c|}\hline
        Browser & Version \\\hline\hline
        Google Chrome & All versions  \\\hline
        Microsoft Internet Explorer & 7.x+ \\\hline
        Mozilla Firefox & 3.7+ \\\hline
        Apple Safari & 5.x+ \\\hline
    \end{tabular}
    \caption{Supported Browsers}
    \label{browsers}
\end{table}

\end{document}