%
% srs.tex contains the Software Requirements Specification for the
%     Software Development Folder for the eCommerce CMS project for:
% CMSI 402
% Loyola Marymount University
%
% By Andrew Won
%

\documentclass{article}
\usepackage{doc}
\usepackage{geometry}
\usepackage{amsmath}
\geometry{letterpaper}

\title{eCommerce CMS Software Requirements Specification}
\author{Andrew Won}
\date{February 4, 2013}

\newcommand{\br}{\vspace{2mm}}

\begin{document}

\maketitle

\abstract{
eCommerce CMS is a software solution for small business customers looking for
an initial online presence.  The eCommerce CMS application will offer both a
web service for HTTP requests and sets of internet-accessible web pages served
by that web service.

This Software Requirements Specification is intended to outline the necessary
outcomes and performance specifications required for eCommerce CMS to be
considered fully functional and deployable.
}

\pagebreak
\tableofcontents
\listoffigures
\listoftables

\pagebreak
\setcounter{section}{4}
\section{Software Requirements Specification}

\subsection{Introduction}

\subsubsection{Identification}

This Software Requirements Specification (SRS) documents the requirements for
the small business utility web service and web site, called eCommerce CMS.

\subsubsection{System Overview}

The eCommerce CMS application will offer a Java-based web service which will
allow for HTTP requests to qualified users.  The Java-based service will also
serve two sets of front-end web pages.  Through the first set, small business
users of the application can implement and manage an online business management
and sales solution. Through the second set, customers of the small business will
be able to interact with the small business either through making purchases on
an eCommerce web site or through committing information required for generation
of a proposal.

\subsubsection{Document Overview}

This SRS is a minimized form of a typical SRS due to time constraints.  In the
following sections there will be a section dedicated to Functional Requirements,
Performance Requirements, and finally Environmental Requirements.  Following these
Requirements sections there will be an appendix.

Table~\ref{terms} defines key terms that are used within this document.

\begin{table}
\begin{tabular}{|l|p{11cm}|}\hline
Term & Definition \\\hline\hline
Shall & A contractual obligation without which this software shall be considered incomplete. \\\hline
Should & An optional provision offering guidance for functional requirements. \\\hline
Will & An optional provision offering guidance for design requirements. \\\hline
User & Small business owner, or agent thereof, intending to deploy a web service
and corresponding web sites. \\\hline
Customer & A customer of the small business owner who intends to use the web service
or one of the web sites deployed by the user. \\\hline
Application & The eCommerce CMS software suite outlined in this SRS. \\\hline
Back-End & The web service offering HTTP endpoints and serving web sites \\\hline
User Front-End & The collection of web sites that offer creation and management
functionality to the user. \\\hline
Customer Front-End & The collection of web sites customized by the user displaying
a web store or data entry fields for proposal generation.\\\hline
\end{tabular}
\caption{Definitions used in this SRS.}
\label{terms}
\end{table}

The application is the Computer Software Configuration Item (CSCI) and consists
of three major Computer Software Components (CSC).  The CSCI is broken down
into the back-end CSC, the user front-end CSC, and the customer front-end
CSC.  Each respective CSC can further be broken down to several Computer
Software Units (CSU).  This document will specify some of the CSU's that are
intended in table~\ref{software-hierarchy}, but this list shall in no way bind
or limit the number of CSU's that will be developed within eCommerce CMS.

\begin{table}
\begin{tabular}{|c|c|c|}\hline
eCommerce CMS & Back-End & \\\hline
\end{tabular}
\caption{CSCI, CSC, and CSU Hierarchy}
\label{software-hierarchy}
\end{table}

\subsection{Functional Requirements}

The Functional Requirements section is segmented into subsections consisting of
one functional requirement each.  These subsections flow sequentially from
back-end, to user front-end, and finally to customer front-end.

\subsubsection{Functional Back-End Requirement 1}
\label{func-back-end-1}

The back-end shall provide a URI contract with a set of HTTP endpoints
to access web service functions for all functions offered by the web service
when hit by an HTTP request.

\subsubsection{Functional Back-End Requirement 2}
\label{func-back-end-2}

The back-end shall respond to all front-end page requests through HTTP endpoints
within 2 seconds in accordance with functional back-end
requirement~\ref{func-back-end-1}.

\subsubsection{Functional Back-End Requirement 3}
\label{func-back-end-3}

The back-end shall provide a persistent data store with saved input data from
the user and from customers to offer invoice, proposal, report, and web store
content generation by the user.

\subsubsection{Functional Back-End Requirement 4}
\label{func-back-end-4}
%    [agent] shall <perform> [FUNCTION] with [Performance] <parameter(s)> while
%    [CONDITIONS] <upon input(s)> in accordance with [INTERFACES].
The back-end shall provide an endpoint that generates

\subsubsection{Functional User Front-End Requirement 1}
\label{func-user-front-end-1}


\subsubsection{Functional Customer Front-End Requirement 1}
\label{func-cust-front-end-1}


\subsection{Performance Requirements}
\subsubsection{Performance Requirement 1}
\subsection{Environment Requirements}
\subsubsection{Development Environment Requirements}
\subsubsection{Execution Environment Requirements}
\end{document}