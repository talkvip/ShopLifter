%
% srs.tex contains the Software Requirements Specification for the
%     Software Development Folder for the eCommerce CMS project for:
% CMSI 402
% Loyola Marymount University
%
% By Andrew Won
%

% ~~~~~~~~~~~~~~~~~~~~~~~~~~~~~~~~~~~~~~~~~~~~~~~~~~~~~~~~~~~~~~~~~~~~~~~~~~~~~
%  Revision History:
%  -----------------
%
%   Ver      Date      Modified by:  Description of change/modification
%  -----  -----------  ------------  ------------------------------------------
%  1.0.0  4-Feb-2013   A. Won        Initial version/release
%  1.0.1  10-Feb-2013  A. Won        Revised per professor instruction
%                                    (more functional break down)
%
% ~~~~~~~~~~~~~~~~~~~~~~~~~~~~~~~~~~~~~~~~~~~~~~~~~~~~~~~~~~~~~~~~~~~~~~~~~~~~~

\documentclass{article}
\usepackage{doc}
\usepackage{geometry}
\usepackage{amsmath}
\usepackage{multirow}
\geometry{letterpaper}

\title{eCommerce CMS Software Requirements Specification}
\author{Andrew Won}
\date{February 4, 2013}

\newcommand{\br}{\vspace{2mm}}

\begin{document}

\maketitle

\abstract{
eCommerce CMS is a software solution for small business customers looking for
an initial online presence.  The eCommerce CMS application will offer both a
web service for HTTP requests and sets of internet-accessible web pages served
by that web service.

This Software Requirements Specification is intended to outline the necessary
outcomes and performance specifications required for eCommerce CMS to be
considered fully functional and deployable.
}

\pagebreak
\tableofcontents
\listoftables

\pagebreak
\setcounter{section}{4}
\section{Software Requirements Specification}

\subsection{Introduction}

\subsubsection{Identification}

This Software Requirements Specification (SRS) documents the requirements for
the small business utility web service and web site, called eCommerce CMS.

\subsubsection{System Overview}

The eCommerce CMS application will offer a Java-based web service which will
allow for HTTP requests to qualified users.  The Java-based service will also
serve two sets of front-end web pages.  Through the first set, small business
users of the application can implement and manage an online business management
and sales solution. Through the second set, customers of the small business will
be able to interact with the small business either through making purchases on
an eCommerce web site or through committing information required for generation
of a proposal.

The eCommerce CMS application is the Computer Software Configuration Item
(CSCI) and consists of three major Computer Software Components (CSC).  The
CSCI is broken down into the back-end CSC, the user front-end CSC, and the
customer front-end CSC.  Each respective CSC can further be broken down to
several Computer Software Units (CSU).  This document will specify some of the
CSU's that are intended in table~\ref{software-hierarchy}, but this list shall
in no way bind or limit the number of CSU's that will be developed within
eCommerce CMS.

\begin{table}
    \begin{tabular}{|c|l|p{8.5cm}|}\hline
        CSCI & CSC & CSU \\\hline\hline
        \multirow{12}{*}{eCommerce CMS}
         & Back-End & Persistent Data Store (Sections~\ref{func-back-end-2} and~\ref{func-back-end-3}) \\\cline{2-3}
         & Back-End & Report/Proposal Generator (Sections~\ref{func-back-end-4} and~\ref{func-back-end-5}) \\\cline{2-3}
         & Back-End & Front-End Generator (Sections~\ref{func-back-end-6} and~\ref{func-back-end-7}) \\\cline{2-3}
         & Back-End & Transaction Processor (Sections~\ref{func-back-end-8} and~\ref{func-back-end-9}) \\\cline{2-3}
         & User Front-End & Visual Web Page Editor (Section~\ref{func-user-front-end-2}) \\\cline{2-3}
         & User Front-End & Web Page Template Editor (Section~\ref{func-user-front-end-4}) \\\cline{2-3}
         & User Front-End & Inventory Input/Edit Form (Section~\ref{func-user-front-end-5}) \\\cline{2-3}
         & User Front-End & Report/Proposal Requestor (Sections~\ref{func-user-front-end-6} and~\ref{func-user-front-end-7}) \\\cline{2-3}
         & Customer Front-End & Proposal Specification Form (Section~\ref{func-cust-front-end-3}) \\\cline{2-3}
         & Customer Front-End & Web Store (Sections~\ref{func-cust-front-end-4} and~\ref{func-cust-front-end-5}) \\\cline{2-3}
         & Customer Front-End & Purchasing/Transaction System (Sections~\ref{func-cust-front-end-6},~\ref{func-cust-front-end-7},~\ref{func-cust-front-end-8},~\ref{func-cust-front-end-9}, and~\ref{func-cust-front-end-10}) \\\hline
    \end{tabular}
    \caption{CSCI, CSC, and CSU Hierarchy}
    \label{software-hierarchy}
\end{table}

\subsubsection{Document Overview}

This SRS is a minimized form of a typical SRS due to time constraints.  The
remainder of this document is structured as follows.  Section 5.2 contains
Functional Requirements, Section 5.3 contains Performance Requirements, and
finally Section 5.4 contains Environmental Requirements.  Following these
Requirements sections there will be an appendix.

Table~\ref{terms} defines key terms that are used within this document.

\begin{table}
    \begin{tabular}{|l|p{11cm}|}\hline
        Term & Definition \\\hline\hline
        Shall & A contractual obligation without which this software shall be considered incomplete. \\\hline
        Should & An optional provision offering guidance for functional requirements. \\\hline
        Will & An optional provision offering guidance for design requirements. \\\hline
        User & Small business owner, or agent thereof, intending to deploy a web service
        and corresponding web sites. \\\hline
        Customer & A customer of the small business owner who intends to use the web service
        or one of the web sites deployed by the user. \\\hline
        Application & The eCommerce CMS software suite outlined in this SRS. \\\hline
        Back-End & The web service offering HTTP endpoints and serving web sites \\\hline
        User Front-End & The collection of web sites that offer creation and management
        functionality to the user. \\\hline
        Customer Front-End & The collection of web sites customized by the user displaying
        a web store or data entry fields for proposal generation.\\\hline
    \end{tabular}
    \caption{Definitions used in this SRS.}
    \label{terms}
\end{table}

\pagebreak
\subsection{Functional Back-End (FBE) Requirements}

The application will serve eCommerce web pages allowing inventory data to be
manipulated and transacted with by the user and customers, respectively.

The Functional Back-End (FBE) Requirements section is segmented into subsections
consisting of one functional back-end requirement each.

\subsubsection{FBE Requirement 1}
\label{func-back-end-1}

The back-end shall provide a URI contract with a set of HTTP endpoints
to access web service functions for all functions offered by the web service
when hit by an HTTP request.

\subsubsection{FBE Requirement 2}
\label{func-back-end-2}

The back-end shall provide a persistent data store.

\subsubsection{FBE Requirement 3}
\label{func-back-end-3}

The back-end persistent data store shall hold any data input by either the
user or by customers.

The back-end persistent data store will offer invoice, proposal, report, and web
store content generation by the user.

\subsubsection{FBE Requirement 4}
\label{func-back-end-4}

The back-end shall provide an endpoint that generates a report based on the
selection and fields provided by the user in the body of the request through
the method outlined in section~\ref{func-user-front-end-8}.

\subsubsection{FBE Requirement 5}
\label{func-back-end-5}

The back-end shall provide an endpoint that generates a proposal based on the
selection and fields provided by the user in the body of the request through
the method outlined in section~\ref{func-user-front-end-8}.

\subsubsection{FBE Requirement 6}
\label{func-back-end-6}

The back-end shall provide a user front-end generator that deploys a set of
documents written in standard HTML.

The front-end deployed by the back-end will allow for management of the
eCommerce site when deployed by the user.

\subsubsection{FBE Requirement 7}
\label{func-back-end-7}

The back-end shall provide a customer front-end generator.

The customer front-end generator will deploy a set of documents written in
standard HTML with customization based on past user input, by the methods
outlined in section~\ref{func-user-front-end-2}, when deployed by HTTP endpoint.

\subsubsection{FBE Requirement 8}
\label{func-back-end-8}

The back-end shall provide a transaction processor, or connection to a
third-party transaction processor, to accept purchasing by customers
when a customer completes a check-out process.

\subsubsection{FBE Requirement 9}
\label{func-back-end-9}

The back-end shall provide a transaction processor.

The transaction processor will log data pertaining to either a sale of inventory
or a generation of a proposal for use in report generation, as outlined in
sections~\ref{func-back-end-4} and~\ref{func-back-end-5}, when a transaction
is conducted.

\pagebreak
\subsection{Functional User Front-End (FUFE) Requirements}

The User Front-End will provide inventory, web site, and report management
functionality.  The user will have the ability to add and edit products and
services that are available to customers and to reports that are generated.  The
user will also manipulate the display of the customer's front-end through a
user interface editor.  The user will be able to edit the layout of reports and
proposals generated by the application and to generate those documents upon
request.

The Functional User Front-End (FUFE) Requirements section is segmented into
subsections consisting of one functional user front-end requirement each.

\subsubsection{FUFE Requirement 1}
\label{func-user-front-end-1}

The user front-end shall provide all web pages in HTML.

The user front-end will allow users to interact with the back-end when opened in
a browser by a user.

\subsubsection{FUFE Requirement 2}
\label{func-user-front-end-2}

The user front-end shall provide a What-You-See-Is-What-You-Get (WYSIWYG) style
editor.

The WYSIWYG style editor will allow a user to edit the layout and appearance of
the customer front-end when the customer front-end is re-deployed by the user,
per Section~\ref{func-user-front-end-3}.

\subsubsection{FUFE Requirement 3}
\label{func-user-front-end-3}

The user front-end shall provide a button to re-deploy the customer front-end
adhering to saved customizations when clicked by the user.

\subsubsection{FUFE Requirement 4}
\label{func-user-front-end-4}

The user front-end shall provide a set of templates that can be deployed to
full-scale customer front-ends with minimal input from the user when an option
is chosen from the user front-end initial home screen.

\subsubsection{FUFE Requirement 5}
\label{func-user-front-end-5}

The user front-end shall provide an inventory management form.

The inventory management form will allow users to input or edit inventory data
as necessary when the user inputs information and clicks a submit button.

\subsubsection{FUFE Requirement 6}
\label{func-user-front-end-6}

The user front-end shall provide a history of transaction details when the user
requests a report through a request page.

The history of transaction details will be in the form of a customizable report.

\subsubsection{FUFE Requirement 7}
\label{func-user-front-end-7}

The user front-end shall provide a document proposal generator.

The document proposal generator will take products or services in the inventory
and aggregate them into a document template when requested by a user through a
request page.

\subsubsection{FUFE Requirement 8}
\label{func-user-front-end-8}

The user front-end document proposal generator shall provide user selection to
select output generation

\pagebreak
\subsection{Functional Customer Front-End (FCFE) Requirements}

The Customer Front-End will provide access to the e-store customized by the user
and provide the ability to purchase or request inventory.  The customer will
be able to setup an account, browse inventory, browse other web sites generated
by the user, and to purchase through a third-party payment processor.

The Functional Customer Front-End (FCFE) Requirements section is segmented into
subsections consisting of one functional customer front-end requirement each.

\subsubsection{FCFE Requirement 1}
\label{func-cust-front-end-1}

The customer front-end shall provide all web pages in HTML.

\subsubsection{FCFE Requirement 2}
\label{func-cust-front-end-2}

The customer front-end shall allow customers to interact with the
back-end and conduct transactions when opened in a browser by a customer.

\subsubsection{FCFE Requirement 3}
\label{func-cust-front-end-3}

The customer front-end shall provide a proposal data submission form.

The data submission form will transmit data that will be used in proposal
generation by the user when a customer opens the form web page and submits the
data by a form submission button.

\subsubsection{FCFE Requirement 4}
\label{func-cust-front-end-4}

The customer front-end shall provide a web store with a product browsing page.

The product browsing page(s) will show key details about products.

\subsubsection{FCFE Requirement 5}
\label{func-cust-front-end-5}

The customer front-end shall provide a web store with a set of product detail
viewing pages.

Any one product detail viewing page will show specific details about a single
product while a customer is on that page.

\subsubsection{FCFE Requirement 6}
\label{func-cust-front-end-6}

The customer front-end shall provide an ``Add to shopping cart'' button on pages
where products or services are displayed, see
sections~\ref{func-cust-front-end-4} and~\ref{func-cust-front-end-5}.

\subsubsection{FCFE Requirement 7}
\label{func-cust-front-end-7}

The customer front-end shall provide a persistent shopping cart.

\subsubsection{FCFE Requirement 8}
\label{func-cust-front-end-8}

The customer front-end's persistent shopping cart shall take items added by the
method described in section~\ref{func-cust-front-end-6}.

The persistent shopping cart will store added items and display them when the
customer goes to a checkout section.

\subsubsection{FCFE Requirement 9}
\label{func-cust-front-end-9}

The customer front-end shall provide a ``Checkout'' button that takes customers
to a web page dedicated to transacting sales of prodcuts; see
section~\ref{func-cust-front-end-10}

\subsubsection{FCFE Requirement 10}
\label{func-cust-front-end-10}

The customer front-end shall provide a web page dedicated to transacting sales
of products in the shopping cart; see section~\ref{func-cust-front-end-8}.

The customer front-end's page dedicated to transacting sales of products will be
displayed when a customer presses a button navigating to the ``Checkout'' page,
by the method specified in section~\ref{func-cust-front-end-9}.

\subsubsection{FCFE Requirement 11}
\label{func-cust-front-end-11}

The customer front-end shall be displayed in the format specified by the user.

The user will specify the format of the customer front-end either through
template selection or through customization through a WYSIWYG,
see sections~\ref{func-user-front-end-2} and~\ref{func-user-front-end-4}.

\pagebreak
\subsection{Performance Requirements}

\subsubsection{Performance Requirement 1}
\label{perf-back-end-1}

The back-end shall respond to all front-end page requests through HTTP endpoints
within 2 seconds in accordance with functional back-end
requirement~\ref{func-back-end-1}.

\subsubsection{Performance Requirement 2}
\label{perf-back-end-2}

The back-end shall respond to requests for reports or proposals through HTTP
endpoints within 10 seconds in accordance with functional back-end
requirements~\ref{func-back-end-4} and~\ref{func-back-end-5} when a user
triggers the process through the methods described in
requirements~\ref{func-user-front-end-6} and~\ref{func-user-front-end-7}.

\subsubsection{Performance Requirement 3}
\label{perf-back-end-3}

The back-end's front-end generator shall respond to a re-deploy request by a
user within 15 seconds in accordance with functional back-end
requirement~\ref{func-back-end-7} when a user triggers the process through the
methods described in requirement~\ref{func-user-front-end-3}.

\subsubsection{Performance Requirement 4}
\label{perf-back-end-4}

The back-end shall respond to requests to complete transactions within 5 seconds
plus any additional time required by third-party vendors in accordance with
functional back-end requirement~\ref{func-back-end-8} and completing logging of
the transaction detailed in requirement~\ref{func-back-end-9}.

\subsubsection{Performance Requirement 5}
\label{perf-back-end-5}

The web pages of both the user and customer front-ends shall be displayed within
3 seconds in accordance with functional requirements~\ref{func-user-front-end-1}
and~\ref{func-cust-front-end-1}.

\pagebreak
\subsection{Environment Requirements}

\subsubsection{Development Environment Requirements}

The application is developed with Java 1.6+ with packages provided by
Maven.  There are no specific system requirements for the development of
the application besides the ability to use Java and the packages provided
by Maven.

\subsubsection{Execution Environment Requirements}

The back-end is served from a web server or from a hosting solution provider and is
required to have the minimum specifications outlined in table~\ref{server}.  The
server itself requires minimum hardware support, but does require an always-on
network connection through which to serve the service and web pages.  A Java
Development Kit (JDK) of version 1.5 or newer is required to compile and package
the Maven repository that will be used.

The front-end is served as web pages within supported web browsers, as listed in
table~\ref{browsers}.  This does not constitue a complete list of supported web
browsers, but is a brief list of web browsers that are guaranteed to be supported
by the web sites.

\begin{table}
    \centering
    \begin{tabular}{|c|c|}\hline
        Category & Requirement \\\hline\hline
        JDK & 1.5+ \\\hline
        Hard Drive Space & 250mb \\\hline
        Operating System & No requirement \\\hline
        RAM & 512mb \\\hline
    \end{tabular}
    \caption{Server Requirements}
    \label{server}
\end{table}

\begin{table}
    \centering
    \begin{tabular}{|c|c|}\hline
        Browser & Version \\\hline\hline
        Google Chrome & All versions  \\\hline
        Microsoft Internet Explorer & 7.x+ \\\hline
        Mozilla Firefox & 3.7+ \\\hline
        Apple Safari & 5.x+ \\\hline
    \end{tabular}
    \caption{Supported Browsers}
    \label{browsers}
\end{table}

\end{document}